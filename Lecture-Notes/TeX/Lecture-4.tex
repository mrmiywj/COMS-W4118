\documentclass[]{article}
\usepackage[all]{xy}
\usepackage{amsmath}
\usepackage{amssymb}
\usepackage{amsthm}
\usepackage{enumitem}
\usepackage{indentfirst}
\usepackage{listings}
\usepackage{multirow}
\usepackage{tikz}
\usepackage{tikz-qtree}
\usepackage{tipa}
\newcommand{\code}{\texttt}
\begin{document}

\newtheorem{thm}{Theorem}
\title{Operating Systems \\ COMS W4118 \\ Lecture 4}
\author{Alexander Roth}
\date{2015 -- 02 -- 03}
\maketitle

\section*{\code{signal}}
\begin{itemize}
\item \code{signal} is the typical introduction to asynchronous operation in the
kernel.
\item \code{signal} takes two arguments, which signal it is, and a pointer to a
function.
\item The function pointer takes one interger and returns nothing.
\item \code{signal} returns \code{SIG\_ERR} if something goes wrong.
\item The signal handler can be any function that takes one integer and returns
nothing.
\item The return value of \code{signal} is the previous signal handler.
\item We include the \code{perror} block for when we register our signal
handler.
\item Signal Handler must be written in a very careful manner so it does not
interrupt what is currently happening in the machine.
\item \code{printf} is a non-reentrant function.
\item Signal handlers are concept that belong to processes, not functions.
\item \code{kill} sends a signal to a process.
\item You can catch any signal and not die.
\end{itemize}

\section*{X Session}
\begin{itemize}
\item When you boot up a virtual console, you have to start xfce4
\item \code{xinit} reads the window configurations and overall configurations
for the machine.
\item \code{Xorg.bin} takes control of the graphics card and draw in the
objects.
\item Old name for \code{Xorg} was just \code{X}.
\item X windows system controls the graphical interface.
\item The terminal program is a program that makes a connection to the X server.
It is a network application.
\item Google X forwarding.
\item Window manager is responsible for resizing windows and all that jazz.
\item Things are pretty modular in unix systems.
\item Remember to take snapshots and how to work with the virtual box.
\item Key idea: Having a layer of systems versus having direct access to the
system.
\end{itemize}

\section*{Kernel Modules}
\begin{itemize}
\item \code{insmod} - inserts modules into the system.
\end{itemize}

\section*{File System}
\begin{itemize}
\item Equivalent system calls to the user-space function calls.
\item \code{open} is system; \code{fopen} in the user-space.
\item \code{mode} is the permissions for the file.
\item \code{fopen} does not set the permission explicitly (i.e., there is no
parameter that needs to be set.
\item \code{fopen} is system dependent. It is not a system call, it's a cross-
platform library.
\item \code{mode} is a very unix-specific method to control file permissions.
\item \code{O\_EXCL} with \code{O\_CREAT} means ``please create this file and
only create this file if this file is not there.''
\item Guarantees mutual exclusion to a given file.
\end{itemize}

\end{document}
