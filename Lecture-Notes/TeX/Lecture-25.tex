\documentclass[]{article}
\usepackage[all]{xy}
\usepackage{amsmath}
\usepackage{amssymb}
\usepackage{amsthm}
\usepackage{enumitem}
\usepackage{indentfirst}
\usepackage{listings}
\usepackage{multirow}
\usepackage{tikz}
\usepackage{tikz-qtree}
\usepackage{tipa}
\begin{document}

\newtheorem{thm}{Theorem}
\title{Operating Systems \\ COMS W4118 \\ Lecture 25}
\author{Alexander Roth}
\date{2015 -- 04 -- 30}
\maketitle

\section{Memory Management 2 Electric Boogaloo: Virtual Memory}
\begin{itemize}
\item If multiple processes use a lot of memory and use more memory than you
have, what do you do?
\item What if you run out of all physical memory because you were running too
many processes?
\item To be able to load more processes and have them use more memory physically
available.
\item Memory access tends to be temporal and spatial.
\item Chances are we need to access memory that is adjacent to each other.
\item A program tends to access a memory address locally for some time and then
jump to different locations.
\item A virtual memory system tries to have a disk as the backing storage for
memory to give the illusion that accessing memory is handled quickly.
\item Page fault refers to the condition where there is a legitimate access to a
page, but the page has been swapped out to a disk before it was accessed.
\item If a page fault occurs, there is a page fault hanlder that halts the
running process, loads the page to memory, and then restart the process that
caused the page fault.
\item From the process point of view, it never knew what happened. It was just
interrupted.
\item When we look for the page, we have to translate the virtual mapping to the
physical mapping.
\item The operating system must know when to bring pages from disk to memory.
\item If you are bringing back something from disk drive, then the chances are
you don't have a free frame.
\item Thus, one of the frames in the table needs to be removed and saved to
disk.
\item This is known as page replacement.
\end{itemize}

\subsection{Page Selection}
\begin{itemize}
\item Load pages when they must absolutely be in memory.
\item Also used when not necessarily need to save memory.
\item The idea is to be as lazy as possible.
\item The user can specify which pages are needed.
\item This is known as request paging.
\item The operating system tends to discount whatever the user asks for.
\item Pre-paging tries to smooth out bursts of page faults within the page
select.
\item A working set refers to all the pages that the processes touches within a
given time.
\item The point of pre-paging, adaptive paging, is to smooth out some of the
page fault rate.
\item If we read pages in at a predictive fashion, we won't have the issue of
and delay of page faults.
\item Thrashing refers to the case where you have a limited amount of RAM but
you are running a large amount of processes.
\item You don't have enough memory for all the processes so the operating system
has to keep saving to disk to make room for the processes.
\end{itemize}

\subsection{Page Replacement Algorithms}
\begin{itemize}
\item Throw out the pages that won't be used for the longest time in the future.
\item The problem is we don't know how long a page will not be used for.
\item So we just have to make an approximation.
\item Every process algorithm has FIFO, so we could use it for the replacement
algorithm.
\item The theory is that the past will produce the future.
\item When we make a decision to choice a frame to save to disk, LRU will look
at the page that was last used the longest.
\item Chances are the algorithm will not be used in the future as well.
\item However, we need to keep track of when everything was last used, which is
expensive.
\item The biggest problem about FIFO is that it looks at which page got loaded
first, which is not a fair statement.
\item What if the item being kicked out is a global variable that is being
accessed all the time?
\item FIFO ignores access patterns.
\item LRU stands for Least-Recently-Used.
\item In a normal program where memory access is localized, LRU is pretty
standard.
\end{itemize}

\subsection{Least Recently Used Algorithm}
\subsubsection{Implementation 1}
\begin{itemize}
\item Hard to implement
\item Need to maintain what was the last item accessed.
\item Every time page is referenced, save system clock into the counter of the
page.
\item Page replacement: scan through pages to find the one with the oldest
clock.
\item Linear scan through all the frames.
\item Every time there is a memory access, the clock field in the entry has to
be updated.
\end{itemize}

\subsubsection{Implementation 2}
\begin{itemize}
\item Every time a frame is referenced, move it to the front of the list that is
containing the pages.
\item The front of the list has the pages that were not accessed in quiet some
time.
\item Manipulating a link list everytime there is a memory access can be pretty
heavy.
\end{itemize}

\subsection{Linux LRU}
\begin{itemize}
\item Linux approximates LRU by not finding the least recently accessed page,
but by finding a not recently accessed page.
\item The page being selected may not be the least recently accessed page, but
it's close.
\item This is known as the clock algorthim or the second-chance algorithm.
\item There is a reference bit within the page entry.
\item Everytime the page is accessed, the bit is turned on.
\item Whenever you read or write something from this page, the reference bit is
turned on.
\item The hardware sets the bit from 0 to 1.
\item The OS tries to find a page with the reference bit cleared.
\item The OS traverse all pages, clearing bits over time.
\item The hardware will keep turning on the reference bit whenever it is
accessed.
\item This system will act like a FIFO but only kick out a page if its reference
bit is not set.
\item The operating system goes through the circularly linked array of pages
looking for whichever have their referenced bits set. It then turns the bit off.
\end{itemize}

\end{document}
