\documentclass[]{article}
\usepackage[all]{xy}
\usepackage{amsmath}
\usepackage{amssymb}
\usepackage{amsthm}
\usepackage{enumitem}
\usepackage{indentfirst}
\usepackage{listings}
\usepackage{multirow}
\usepackage{tikz}
\usepackage{tikz-qtree}
\usepackage{tipa}
\begin{document}

\newcommand{\code}{\texttt}
\newtheorem{thm}{Theorem}
\title{Operating Systems \\ COMS W4118 \\ Lecture 15}
\author{Alexander Roth}
\date{2015 -- 03 -- 12}
\maketitle

\section{Spin Locks}
\begin{itemize}
\item Checks a signal variable over and over.
\item Wastes CPU cycles.
\end{itemize}

\subsection{Using \code{yield()} with Sleep Locks}
\begin{itemize}
\item We can replace the empty \code{while} loop with a call to the function
\code{yield}.
\item \code{Yield} is not defined.
\item There must be some notion of a variable and we must let the operating
system put other processes to sleep.
\item The operating system must have some way of knowing who is waiting on which
variable.
\item We need some data structure that keeps track of which processes are
waiting on the lock.
\item If the operating system were to wake up all the processes at once, there
would be a huge overhead from all the context switches and only one processes
would get the lock.
\item There is no knowing when we will get a turn.
\item Thus, \textbf{starvation} is a real possibility.
\item Bounded waiting is not satisfied.
\item There must be some notion of a \code{wake} function and some idea of a
queue that contains the processes.
\item Each lock will have a queue associated with it that holds all the waiting
processes.
\item Thus, there are processes waiting in a first come, first serve fashion.
\item Having a \code{flag} variable has the notion of a lost wakeup and a
starved process.
\item Starvation is the hardest condition to satisfy with locking.
\item Oftentimes, the condition will be sacrificed in order to get the system
working.
\end{itemize}

\subsubsection{Fixing Sleeping Locks}
\begin{itemize}
\item \code{mutex} structure that has a flag and a queue.
\item \code{Lock}: function acquires a lock by setting the flag, otherwise it
puts itself into the queue.
\item \code{Unlock}: if queue is empty, set the flag ot zero or dequeue and
wakeup the process we took out of the queue.
\item Flag does not get reset when we take a process off of the queue.
\item To avoid losing wake ups, we implement a spin lock with a guard variable.
\item Inside the \code{yield} function, we need to check that we are dequeued
already.
\item You need to recognize when a race condition is possible.
\item You need to make the area that holds the spin lock as small as possible.
\item Spin lock must happen in the kernel, and should only happen in the cases
where you are doing something very short.
\end{itemize}

\section{Readers-Writers Problem}
\begin{itemize}
\item Old, but common paradigm. You have to read a lot, but sometimes you have
to write.
\item We have a single lock and we lock and unlock in different modes.
\item Lock and unlock in either read or write mode.
\end{itemize}

\subsection{Implementing Reader-Writer Lock}
\begin{itemize}
\item For read lock, we have a counter that notes how many readers lock at a
time.
\item When one reader has a lock, all the other readers will keep going.
\item When you increment the number of readers, we need to lock and unlock while
we increment to maintain atomicity.
\item Synchronization is not easy.
\item The issue with this implementation is that the writer may starve if there
is a constant stream of readers.
\item Writer can only activate until all the readers have moved through.
\item We introduce another lock called \code{writer}, that locks around a
\code{write\_lock} call in order to acquire the original lock.
\item Using one lock for two purposes is typically a bad idea.
\end{itemize}

\section{Semaphore Motivation}
\begin{itemize}
\item There are some things a simple \code{mutex} cannot do.
\item Sometimes you want two threads, one to prepare a data structure and one to
wait until the first is done.
\item You want to impose order.
\item Mutexes do not provide order; semaphores do.
\item Semaphores can handle the producer/consumer problem.
\item A semaphore is some synchronization variable that contains an integer
value.
\item The integer value cannot be accessed directly.
\item Binary semaphore is similar to a mutex, while a counting semaphore has an
integer value more than 1.
\item You can post over the initial value of the semaphore.
\item The initial value is not the maximum value; you can extend past it.
\end{itemize}

\subsection{Producer-Consumer Problem}
\begin{itemize}
\item We have a fixed-bound buffer that contain a mix of data and nothing.
\item Producer has a pointer to the next empty slot.
\item Consumer has a pointer to the first filled slot.
\item Multiple consumers will consume an item in each slot and move the pointer
to the next slot.
\item If consumer reaches the end of the buffer, it wraps around to the
beginning of the buffer, so the buffer is a circular buffer.
\item The buffer is filled when Producer is right behind the Consumer pointer.
\item The buffer is empty when the Consumer pointer is right behind the Consumer
counter.
\end{itemize}

\subsubsection{Producer-Consumer Implementation}
\begin{itemize}
\item We create two semaphores: \code{full} and \code{empty}.
\item Producer waits on the empty semaphore until empty is non-zero. If there is
no empty spots, then we cannot produce.
\item Once empty becomes a positive number, then there is a slot open in the
array.
\item Producer will then decrement empty, and fill the slot, finally increment
the number of the full semaphore to say there is a newly filled position.
\item Consumer will do the opposite of the producer function.
\item If full is zero, then there is no item in the buffer; otherwise, empty the
slot and increment the empty semaphore.
\item Need to synchronize the movement between the empty and full semaphores.
\item Thus, we create a new guard semaphore during the fill slot and empty slot
phases.
\item Needs to be guarded to avoid multiple consumers and producers creating a
race condition.
\item The guard provides the mutual exclusion property.
\end{itemize}

\section{Monitors and Condition Variables}
\subsection{Montiors}
\begin{itemize}
\item Monitor is the fusion of synchronization and object-oriented programming.
\item Every java object is a monitor.
\item It is an object-oriented class where every method is a synchronized
method.
\item Every method you call has to grab the lock and release said lock at the
end of the method.
\item Every object has hidden variables, which is a mutex and $N$-many
conditional variables.
\end{itemize}

\end{document}
