\documentclass[]{article}
\usepackage[all]{xy}
\usepackage{amsmath}
\usepackage{amssymb}
\usepackage{amsthm}
\usepackage{enumitem}
\usepackage{indentfirst}
\usepackage{listings}
\usepackage{multirow}
\usepackage{tikz}
\usepackage{tikz-qtree}
\usepackage{tipa}
\begin{document}

\newcommand{\code}{\texttt}
\newtheorem{thm}{Theorem}
\title{Operating Systems \\ COMS W4118 \\ Lecture 7}
\author{Alexander Roth}
\date{2015 -- 02 -- 12}
\maketitle

\section{Homework}
\begin{itemize}
\item Build http server. Has 5 parts.
\item Understand the mechanics and understand why they work.
\end{itemize}

\subsection{Part 0 Basic Single-Process/Single-Threaded Web Server}
\begin{itemize}
\item Same HTTP server from AP
\item Only serves one user at a time
\item Read through the code and understand everything
\end{itemize}

\section{POSIX Semaphores}
\begin{itemize}
\item Designed so shared processes don't interfere with each other and avoids a
race condition.
\item The keyword \code{synchronized} in Java causes the method to run
atomically. No other method running on a thread can occur during this method's
call.
\item When this method is called, the hidden \code{lock} on a Java object will
be set. On exit, the \code{lock} will be unset.
\item This is a version of mutual exclusion between locking methods.
\item A \code{semaphore} is similar to a lock, except you can lock multiple
times.
\item Semaphore is initialized with a number, such as 4.
\item Different processes can decrement the number.
\item \code{sem\_post} means that the process is done and increment the number
of the semaphore.
\item \code{sem\_wait} means that the process is entering the semaphore and that
the number of the semaphore should be decremented.
\item A binary semaphore is equivalent to a \code{mutex} lock. Only one process
can hold onto the semaphore now.
\item Locking with a semaphore is a completely voluntary process. There is
nothing that will necessary stop another process from entering the pool unless
it calls \code{sem\_wait}.
\item There is no requirement that the process that grabbed the semaphore number
has to call \code{sem\_post}. Any process can do that.
\item \code{mutex} forces that there is a mutual exclusions between processes.
So if one process locks a part of the program, it must be the process to unlock
that part.
\item There are two kinds of semaphores:
\begin{enumerate}
\item Named semaphores - Can be used between unrelated processes.
\item Unnamed semaphores - Semaphore lives on a mapped area of shared memory.
\end{enumerate}
\item Typical convention for the name of a semaphore is a slash followed by
something relted to the process.
\item The named semaphore function \code{sem\_open} returns a pointer to the
semaphore.
\item Flag value lets the system know if you are creating the semaphore or
opening an existing one.
\item Semaphores are not deallocated until you call \code{sem\_unlink}.
\item Often times, you create a temporary file while the program is running,
when you finish the task, you remove the temp file.
\item To delete this file, you open it and get a file descriptor of it, then you
immediately call link.
\item Once a file loses all of its links, then the file will be removed.
\item The semaphore API looks like the file API because it is backed up by a
file.
\end{itemize}

\section{Signals}
\begin{itemize}
\item One of the most complicated topics in a UNIX system.
\item Concurrency is the main concept behind operating systems.
\item Asynchronous calls are the worst part of concurrency.
\end{itemize}

\subsection{Process Groups, Sessions, Controlling Terminal}
\begin{itemize}
\item Controlling terminal is the terminal program that started the session.
\item A single process, the session leader, interacts with the controlling
terminal in order to ensure that all programs are terminated when a user ``hangs
up'' the terminal connection.
\end{itemize}

\subsection{Sending Signals}
\begin{itemize}
\item \code{kill} command delivers a signal into a process ID.
\item \code{raise} is the same thing as \code{kill} but it commits suicide.
\item \code{raise} is a simple wrapper on top of the \code{kill} function.
\item \code{pid} is not part of C; it is UNIX-specific.
\end{itemize}

\section{SIGALRM}
\subsection{alarm() and pause() functions}
\begin{itemize}
\item Set up a signal handler for \code{SIGALRM}, give it a time, and run it.
\item Programs will block at \code{pause()} command.
\item This is a good way to implement \code{sleep}. Create an alarm signal
handler and set pause.When the timer runs out, alarm will be called and break
out the program.
\item You always call \code{alarm(0)} to cancel an alarm.
\end{itemize}

\subsection{Using setjmp \& longjmp to solve the two problems}
\begin{itemize}
\item Part of the standard C language
\item Across function goto statements
\item \code{setjmp} on the first call will return 0, on the next call it will
return whatever value you gave to \code{longjmp}.
\item It's not easy to deal with race conditions.
\end{itemize}

\end{document}
