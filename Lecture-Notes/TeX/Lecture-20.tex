\documentclass[]{article}
\usepackage[all]{xy}
\usepackage{amsmath}
\usepackage{amssymb}
\usepackage{amsthm}
\usepackage{enumitem}
\usepackage{indentfirst}
\usepackage{listings}
\usepackage{multirow}
\usepackage{tikz}
\usepackage{tikz-qtree}
\usepackage{tipa}
\begin{document}

\newcommand{\code}{\texttt}
\newtheorem{thm}{Theorem}
\title{Operating Systems \\ COMS W4118 \\ Lecture 20}
\author{Alexander Roth}
\date{2015 -- 04 -- 08}
\maketitle

\section{Examining the Kernel Memory}
\begin{itemize}
\item We are now in the process stack for the kernel
\item Suppose we have four processes running.
\item Each process has a task struct that is a linked list connecting it to the
next process.
\item You can only have a fixed number of processes.
\item There is also a \emph{run queue} that links together some of the
processes.
\item If a processes are in the run queue, their state is set to
\code{TASK\_RUNNING}, which is the value of 0.
\item \code{TASK\_RUNNING} means the process is eligible to run, not that it is
currently running.
\item Run queues in 2.6 are organized into a huge number of queues, with a given
priority, so the big O notation is constant time.
\item In 3.x, we use a CFS algorithm (Completely Fair Scheduler). It tries to
give more priority on I/O intensive processes.
\item Think of the run queue as linking all the processes that are able to run.
\end{itemize}

\section{Task States}
\begin{itemize}
\item Each task struct has an integer field called \code{tick}, which is how
many clock ticks it has gone through.
\item We can assign a timeout value to the amount of ticks that a process can go
through.
\item There is a flag within the \code{task\_struct} called
\code{NEED\_RESCHED}, which is set whenever a timeout occurs.
\item Every interrupt has a common routine when it is about to return.
\item The common routine is right before the system goes back in the user mode.
\item We check the \code{NEED\_RESCHED} flag.
\item We call the function \code{schedule()} to switch tasks.
\item Eventually, \code{schedule()} calls a function called
\code{context\_switch()}
\item Context switch is system-dependent, and it saves the current process
information and then go to the next task that was picked to run.
\item We have a list of runnable processes that the system will look through
with the scheduling algorithm to determine which process to run next.
\item Timer interrupts have the ability to preempt other processes.
\item A running process can go to sleep and enter the sleeping state.
\item When does a running process enter into a blocked sleeping state?
\item There are certain situations when the kernel makes processes
uninterruptible.
\item A blocking system call sends the running task into a sleeping process.
\item Suppose we are blocking a task that is reading from a keyboard.
\item The read system call would set the task to interruptible while it waits
for the keyboard.
\item There must be some data structure that knows what process task struct is
waiting on it.
\item That is why there is a wait queue.
\item When the system receive information from the keyboard, the task struct
will be woken up from the wait queue on \code{wake\_up()}
\item We run the command \code{try\_to\_wake\_up()}.
\item When an interruptible process gets a signal, it wakes up but a field is
set to go to the signal handler.
\end{itemize}

\end{document}
