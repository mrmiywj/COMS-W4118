\documentclass[]{article}
\usepackage[all]{xy}
\usepackage{amsmath}
\usepackage{amssymb}
\usepackage{amsthm}
\usepackage{enumitem}
\usepackage{indentfirst}
\usepackage{listings}
\usepackage{multirow}
\usepackage{tikz}
\usepackage{tikz-qtree}
\usepackage{tipa}
\begin{document}

\newcommand{\code}{\texttt}
\newtheorem{thm}{Theorem}
\title{Operating Systems \\ COMS W4118 \\ Lecture 14}
\author{Alexander Roth}
\date{2015 -- 03 -- 10}
\maketitle

\section{How to Implement Locks}
\subsection{Disable Interrupts}
\begin{itemize}
\item Used in early linux kernel when there was only one linux processor.
\item As soon as there were multiple CPU's this method stopped working.
\end{itemize}

\section{Implementing Locks Using Software}
\begin{itemize}
\item Peterson's algorithm works, but not in modern systems.
\end{itemize}

\subsection{First Attempt}
\begin{itemize}
\item Most na\"ive attempt; there is an integer flag and you can check if it is
set.
\item When the flag is zero, someone has unlocked the lock.
\item This method does not work as there will be a gap between the time you set
the lock and the time you check the lock.
\item It is possible that both threads could enter a critical section without
realizing there is another thread.
\item Any time you see a lock that needs a flag to be set, you need to keep an
eye out for a gap in the program where two threads could occur.
\end{itemize}

\subsection{Second Attempt}
\begin{itemize}
\item Two flags, both set to 0.
\item First flag mapped to the first thread and so on.
\item When a thread wants to take a lock, it calls \code{lock}.
\item It then sets the corresponding flag number.
\item It checks the flag previous to it to see if the other thread wants to be
there.
\item As long as the other thread does not want the lock, the current thread can
go forward.
\item You set your flag and then you test the other flag; thus, it eliminates
the critical section race condition.
\end{itemize}

\subsection{Third Attempt}
\begin{itemize}
\item Single variable that indicates who's turn it is.
\item When it is your turn, you run. Otherwise, you wait.
\item Achieves mutual exclusion, only one thread can run at a time.
\item There is no way that two threads can be in a critical section together.
\item The problem is that it can create a deadlock.
\item Three conditions for a lock"
\begin{enumerate}
\item Must provide mutual exclusion
\item Must have each thread progress.
\item There cannot be starvation on a thread.
\end{enumerate}
\item Mutual exclusion is preserved.
\item Thread progress is not preserved.
\item What happens if a thread enters a critical section, grabs a lock, unlocks,
and wants to lock again.
\item However, when you unlock, you must give away your turn.
\item A lock cannot depend on the thread's behavior.
\item This system depends on threads outside of the critical section of the
system.
\end{itemize}

\subsection{Final Attempt: Peterson's Algorithm}
\begin{itemize}
\item Combination of the previous two approaches.
\item Turn-based mechanism AND intent-based mechanisms.
\item Problem with turn-based system was that you must alternate.
\item Here you kind of do the same thing, but first you wait until its the other
guy's turn.
\item It only matters who's turn it is as long as both are interested in using
the system.
\item If both are not interested, then you don't have to wait for the other
thread's turn.
\end{itemize}

\subsection{Memory Barriers}
\begin{itemize}
\item Ensures that all memory operations up to the barrier are executed.
\item Peterson algorithm will not work because of modern CPU behavior.
\item Barrier is used a lot in Linux code.
\end{itemize}

\section{Version 3: Hardware Instructions}
\begin{itemize}
\item Version 1: Disable interrupt
\item Version 2: Software Locks
\item The only practical solution is to have a hardware instruction.
\item Have a static variable called \code{flag}, when \code{test\_and\_set}
operates, return the previous value of the flag.
\item Busy waiting is bad.
\item \code{volatile} keyword means that the memory location pointed to by the
pointer may change behind you.
\item So multi-threaded programs need to know that these memory locations
change.
\end{itemize}

\section{Spin-wait or block?}
\begin{itemize}
\item Problem of spin-wait: waste CPU cycles.
\item You want to avoid sleeping as a sleep command has an overhead attached
with it.
\item Having a spin lock is good for short operations.
\item You only spin when the lock is taken by another thread.
\item Spin locks should be used if the lock is released quickly as long as the
critical section is small.
\item It the thread has a large critical section, it is better to put the thread
to sleep and check after the thread is done.
\item Typically not used for user-programs as user programs can be preempted at
any time.
\item Spin-locks are not useful when you cannot control the preemption of the
system.
\item POSIX has a spinlock API, but it is rarely used.
\item If you are using a uni-processor, it's a bad idea to use a spin-lock
becuase it will spin forever.
\item There is no reason to spin if you are not going to get interrupted.
\end{itemize}

\section{Yield}
\begin{itemize}
\item Instead of continuously checking a process, just put the process to sleep
and check at a later time.
\item The operating system will move onto a different process.
\item The issue is that someone has to wake up the threads to check for the
flag.
\item \textbf{Thundering Herd} - all the locks wake up to just have one lock
receive and then the rest will go back to sleep.
\item These processes are just very inefficient; why should all the threads wake
up when only one thread will take the event.
\end{itemize}

\end{document}
