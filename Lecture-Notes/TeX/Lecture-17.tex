\documentclass[]{article}
\usepackage[all]{xy}
\usepackage{amsmath}
\usepackage{amssymb}
\usepackage{amsthm}
\usepackage{enumitem}
\usepackage{indentfirst}
\usepackage{listings}
\usepackage{multirow}
\usepackage{tikz}
\usepackage{tikz-qtree}
\usepackage{tipa}
\begin{document}

\newcommand{\code}{\texttt}
\newtheorem{thm}{Theorem}
\title{Operating Systems \\ COMS W4118 \\ Lecture 17}
\author{Alexander Roth}
\date{2015 -- 03 -- 31}
\maketitle

\section{Linux System Call Dispatch}
\begin{itemize}
\item Suppose we wrote a hello world application that uses \code{printf}.
\item \code{printf} is a library code that is run within your executable.
\item At the end of the day, \code{printf} will call the system call
\code{write}.
\item During the \code{write} call, the system will enter the \code{kernel}.
\item \code{write} is a C function; \code{write} is a library function but we
regard it as a system call because the library function is pretty much the same
as the system call.
\item \code{write} stores the number for a write system call into a register.
\item The instruction \code{int 0x80} calls a system interrupt to the OS.
\item There are many types of software interrupts. \code{0x80} is reserved for
invoking system calls.
\item When you call \code{int 0x80}, we jump to the kernel to the function
called \code{system\_call}.
\item \code{system\_call} runs a function that points to a table called
\code{sys\_call\_table}, which contains the \code{sys\_write} function.
\item Every system call has to have a corresponding C level library wrapper.
\item System calls just invoke an interrupt with a number in a register.
\item Arguments for system calls go into different registers.
\item The act of invoking a system call should not be privileged because there
needs to be a method to get you into the privileged mode.
\end{itemize}

\section{System Call Parameter Passing}
\begin{itemize}
\item In linux, when there are fewer than 6 parameters, they are passed into
register.
\item All linux system calls are either integers or pointer parameters.
\item Linux has 300+ system calls.
\item If you are writing a system call, you have to make sure that whatever
parameter is passed in, must be checked.
\item You cannot trust random pointer values that the user gave you.
\item At the system call level, everything is privileged. Whatever happens will
happen.
\item If you blindly start adding values to a pointer, you could accidentally
corrupt the kernel.
\item You must make sure that whatever something points to points to the
legitimate part of the program.
\item If user passes a parameter, you must call a helper function to pull the
information from the userspace.
\item These helper functions can block becuase you need to allocate memory for
the system.
\item Memory allocation simply means a lot of things can happen in the
background while you are locating a section of memory to access and backup.
\end{itemize}

\section{Tracing System Calls in Linux}
\begin{itemize}
\item \code{strace} used to trace system calls when a progarm is run.
\item \code{strace} is useful for debugging a program.
\item It's impossible to monitor a program without changing the state of the
program.
\item The kernel has to fundamental support tracing in the code in order to
allow for debugging.
\item You can give an environment variable to the main method.
\item \code{brk} is a system call that allocates items onto the heap.
\item \code{malloc} increases the heap region by some fixed amount and manages
the size by some number of bytes.
\item Unless you force it, gcc will not link the C library statically.
\item The C library code is actually loaded dynamically at runtime.
\item Thus, the newest version of the C code can be loaded in at once, and there
can be one copy of the library code.
\item All memory is virtual memory, which map to the physical space underneath.
\end{itemize}

\end{document}
