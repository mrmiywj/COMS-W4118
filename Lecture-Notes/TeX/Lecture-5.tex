\documentclass[]{article}
\usepackage[all]{xy}
\usepackage{amsmath}
\usepackage{amssymb}
\usepackage{amsthm}
\usepackage{enumitem}
\usepackage{indentfirst}
\usepackage{listings}
\usepackage{multirow}
\usepackage{tikz}
\usepackage{tikz-qtree}
\usepackage{tipa}
\newcommand{\code}{\texttt}
\begin{document}

\newtheorem{thm}{Theorem}
\title{Operating Systems \\ COMS W4118 \\ Lecture 5}
\author{Alexander Roth}
\date{2015 -- 02 -- 05}
\maketitle

\section{System Calls}
\subsection{\code{lseek}}
\begin{itemize}
\item Deals with file descriptors, give offset, \emph{whence} controls how to
interpret the offset.
\item Three offset markers \code{SEEK\_SET}, \code{SEEK\_CUR}, and
\code{SEEK\_END}.
\end{itemize}

\subsection{\code{read}}
\begin{itemize}
\item Reads in a file from a buffer using the file descriptor.
\item There is no end of file byte within a file, the file is just straight
content.
\item End Of File is a condition, which points to one past the last byte in the
file.
\item When you call \code{read}, the number of bytes read may be less than the
requested number of bytes.
\item This is mainly due to the size of the file.
\item There is a notion of ``slow system call.'' That is, a system call where it
is possible for the system to block forever.
\item Reading from a disk is not a ``slow system call'', it will take a long
time though.
\item Reading from a keyboard device is an example of a slow system call. It is
waiting for input from a user.
\item \code{read} call is waiting for input from anything, whether it be another
driver or a keyboard.
\item \code{read} listening on file descriptor 0 is listening for input from the
keyboard.
\item \code{pipes} also behave like files. Thus, the incoming input from one
program may never happen, which would cause the second program to block forever.
\item \code{FIFO} are just named pipes, so they behave like \code{pipe}s.
\item For \code{sockets}, we use \code{recv} as it is socket-specific; however,
we can use an equivalent version of \code{read} for it as well.
\item In UNIX, everything is a file. There is a uniform interface of reading and
writing.
\end{itemize}

\subsection{\code{write}}
\begin{itemize}
\item Again returns the number of bytes written to a file, it can be less than
the specified size.
\item Returns -1 on error, otherwise it returns the number of bytes written to
the file.
\item \code{write()} can be slow as well.
\item \code{pipe} involves coordination between the two processes to write into
the \code{pipe} and read from the \code{pipe}.
\item \code{write} in socket is equivalent to \code{send} for sockets.
\item With the \code{O\_APPEND} flag declared, all \code{write}s will
automatically go to the end of the file.
\item \emph{Atomic} operations are operations that cannot be broken apart; they
must happen together or they won't happen at all.
\item Setting the offset and writing to the file happen in an atomic operation
together.
\item It's very difficult to ensure consistency across a Network-File-System. We
say it may be atomic, but there are scenarios where the operation is not atomic.
\end{itemize}

\section{Signals and System Calls Revisted}
\begin{itemize}
\item When you block on a slow system call, a signal will interrupt the system
call and return the system call with an error.
\item Once you catch a signal, it gets reset to the default value. The signal
disposition gets reset to its initial value.
\item This only occurs on slow system calls.
\item When we have a signal that needs to be reset by a signal handler, we have
an issue where there in a brief time where the signal can happen without the
desired functionality.
\item Signal handlers cannot call reentrant functions.
\item When a signal handler gets called, you must use reentrant functions.
\item Non-reentrant functions are functions where in the middle of the function
you cannot stop and have another thread of your program call that function.
\item Because signal is asynchronous, execution is stopped when signal
interrupts.
\item If a function cannot be re-entered in the after an interrupt has occured,
then you cannot use it in a signal interrupt.
\item From a signal handler, call only \emph{async-signal-safe} functions.
\end{itemize}

\section{File Sharing}
\begin{itemize}
\item In order to understand interprocess code, you need to understand
\code{pipe}, in order to understand \code{pipe}, you need to understand file
descriptors.
\item The kernel maintains a process table entry for every process within the
system.
\item There is a file descriptor table, and pointers to files.
\item The pointers point to file table entries.
\item These table entries contain file status flags, current file offset, and
v-node pointer.
\item You can duplicate a file descriptor from another file descriptor using
\code{dup(1)}.
\item \code{v-node} and \code{i-node} are essentially the same thing.
\item When you \code{fork}, each file descriptor entry gets duplicated into the
child process.
\item Child processes share the same offset for a file, so they will not
overwrite each other. They can interleave each other, but they will not
overwrite their data.
\item The process table entries remain the same and are controlled by the
kernel, not the user-space.
\item Signal handlers are lost when you \code{exec} as the program is not the
same memory space.
\end{itemize}

\section{\code{pipe}}
\begin{itemize}
\item Creates two file descriptors, one that reads from the pipe and the other
that writes to the pipe.
\item By linking the \code{pipe} to the same process, we can then fork and have
a connection between the parent and the child process.
\item We can then close the read end of the parent and the write end of the
child to create a half-duplex channel from the parent to the child.
\end{itemize}

\end{document}
