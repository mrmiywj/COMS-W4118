\documentclass[]{article}
\usepackage[all]{xy}
\usepackage{amsmath}
\usepackage{amssymb}
\usepackage{amsthm}
\usepackage{enumitem}
\usepackage{indentfirst}
\usepackage{listings}
\usepackage{multirow}
\usepackage{tikz}
\usepackage{tikz-qtree}
\usepackage{tipa}
\begin{document}

\newcommand{\code}{\texttt}
\newtheorem{thm}{Theorem}
\title{Operating Systems \\ COMS W4118 \\ Lecture 22}
\author{Alexander Roth}
\date{2015 -- 04 -- 16}
\maketitle

\section{Berkeley Fast File System (FFS) Layout}
\begin{itemize}
\item Partition means you have a physical drive and you separate each section of
the drive into a filesystem.
\item Partitions are also known as \emph{Volumes}
\item iNode is a single block per file; each file has an inode. INode keeps
track of the location of data within the file as well as the metadata for the
file.
\item i-node is a fixed array of numbers. Each file is uniquely identified by
the i-node number in the file system.
\item Block bitmap is used to find out which block of memory is not in use.
\item A cylinder is the same track on a disk. A cylinder group is a larger
number of tracks linked together.
\item The basic structure is a list of i-nodes that point to data blocks.
\item You can have multiple filenames pointing to the same file in the UNIX
system.
\end{itemize}

\section{Hard Links Versus Symlinks}
\begin{itemize}
\item Every hard link is equal. It is a link that refers to some i-node number.
\item You can get stuck in a cycle if two hardlinks are connected to each other.
\item You cannot make a hard link to a directory in the unix system.
\item Symbolic link is a file itself.
\item A hard link is an actual entry in the directory, a symbolic link is just a
shortcut.
\item Symbolic links are just names that point to any file.
\item Hardlinks must point to an i-node in the file system.
\item You cannot have two hardlinks from different file systems because that
uses two separate i-node directories.
\end{itemize}

\section{Schematic View of Virtual File System}
\begin{itemize}
\item Regular hard disks can be mounted and unmounted just like mobile drives.
\item The operating system must be able to mount and unmount any file system.
\item VFS was created because the operating system simply writes code using the
interface defined by the VFS. Different types of file systems will implement the
VFS interface.
\end{itemize}

\section{LFS Idea}
\begin{itemize}
\item Let's just treat the file system as a large tape.
\item Everything is sequential, there is no random access.
\item Log structured file system.
\item Since we are doing a lot more reading than writing to a file system, we
should store all files being read sequentially.
\end{itemize}

\end{document}
