\documentclass[]{article}
\usepackage[all]{xy}
\usepackage{amsmath}
\usepackage{amssymb}
\usepackage{amsthm}
\usepackage{enumitem}
\usepackage{indentfirst}
\usepackage{listings}
\usepackage{multirow}
\usepackage{tikz}
\usepackage{tikz-qtree}
\usepackage{tipa}
\begin{document}

\newcommand{\code}{\texttt}
\newtheorem{thm}{Theorem}
\title{Operating Systems \\ COMS W4118 \\ Reading Notes 5}
\author{Alexander Roth}
\date{2015 -- 02 -- 22}
\maketitle

\section*{Advanced Programming in the Unix Environment \\ Chapter 11: Threads}
\subsection*{11.1 Introduction}
\begin{itemize}
\item All threads within a single process have access to the same process
components, such as file descriptors and memory.
\item Anytime you try to share a single resource among multiple users, you have
to deal with consistency.
\end{itemize}

\subsection*{11.2 Thread Concepts}
\begin{itemize}
\item A typical UNIX process can be thought of as having a single thread of
control: each process is doing only one thing at a time.
\item Multi-threading allows us to design programs that can do more than one
thing at a time within a single process, with each thread handling a separate
task.
\item We can simplify code that deals with asynchronous events by assigning a
separate thread to handle each event type.
\item Threads automatically have access to the same memory address space and
file descriptors.
\item The processing of independent tasks can be interleaved by assigning a
separate thread per task. Two tasks can be interleaved only if they don't depend
on teh processing performed by each other.
\item Interactive programs can realize improved response time by using multiple
threads to separate the portions of the program that deal with user input and
output from the other parts of the program.
\item A program can be simplified using threads regardless of the number of
processors, because the number of processors doesn't affect the program
structure.
\item A thread consists of the information necessary to represent an execution
context within a process
\item This includes:
\begin{enumerate}
\item A \emph{thread ID} that identifies the thread within a process
\item A signal mask
\item A set of register values
\item A stack
\item A scheduling priority
\item A policy,
\item A signal mask
\item An \code{errno} variable
\item And thread-specific data
\end{enumerate}
\end{itemize}

\subsection*{11.3 Thread Identification}
\begin{itemize}
\item Every thread has a thread ID, which is only significant within the context
of the process to which it belongs.
\item A thread ID is represented by the \code{pthread\_t} data type.
\item A function must be used to compare two thread IDs.
\end{itemize}

\code{\#include <pthread.h>}

\code{int pthread\_equal(pthread\_t \emph{tid1}, pthread\_t \emph{tid2});}

Returns: nonzero if equal, 0 otherwise

\begin{itemize}
\item A thread can obtain its own thread ID by calling the \code{pthread\_self}
function.
\end{itemize}

\code{\#include <pthread.h>}

\code{pthread\_t pthread\_self(void);}

Returns: the thread ID of the calling thread

\begin{itemize}
\item \code{pthread\_self} and \code{pthread\_equal} can be used when a thread
needs to identify data structures that are tagged with its thread ID.
\end{itemize}

\subsection*{11.4 Thread Creation}
\begin{itemize}
\item As the program runs, its behavior should be indistinguishable from the
traditional process, until it creates more threads of control.
\item Additional threads can be created by calling the \code{pthread\_create}
function.
\end{itemize}

\code{\#include <pthread.h>}

\code{int pthread\_create(pthread\_t *restrict \emph{tidp}, const
pthread\_attr\_t *\emph{attr}, void *(*\emph{start\_rtn})(void *), void
*restrict \emph{arg});}

Returns: 0 if OK, error number on failure.

\begin{itemize}
\item The memory location pointed to by \emph{tidp} is set to the thread ID of
the newly created thread when \code{pthread\_create} returns successfully.
\item The \emph{attr} argument is used to customize varoius thread attributes.
\item The newly created thread starts running at the address of the
\emph{start\_rtn} function.
\item \emph{arg} is a single argument passed to \emph{start\_rtn}.
\item When a thread is created, there is no guarantee which will run first: the
newly created thread or the calling thread.
\item The set of pending signals for the thread is cleared for the newly created
thread.
\end{itemize}

\subsection*{11.5 Thread Termination}
\begin{itemize}
\item  A single thread can exit in three ways, thereby stopping its flow of
control, without terminating the entire process.
\begin{enumerate}
\item The thread can simply return from the start routine. The return value is
the thread's exit code.
\item The thread can be cancelled by another thread in the same process.
\item The thread can call \code{pthread\_exit}.
\end{enumerate}
\end{itemize}

\code{\#include <pthread.h>}

\code{void pthread\_exit(void *\emph{rval\_ptr});}

\begin{itemize}
\item The \emph{rval\_ptr} argument is a typeless pointer, similar to the single
argument passed to the start routine.
\item This pointer is available to other threads in the process by calling the
\code{pthread\_join} function.
\end{itemize}

\code{\#include <pthread.h>}

\code{int pthread\_join(pthread\_t \emph{thread}, void **\emph{rval\_ptr});}

Returns: 0 if OK, error number on failure.

\begin{itemize}
\item The calling thread will block until the specified thread calls
\code{pthread\_exit}, returns from its start routinge, or is canceled.
\item If the thread simply returnf from its start routine, \emph{rval\_ptr} will
contain the return code.
\item If the thread was canceled, the memory location specified by
\emph{rval\_ptr} is set to \code{PTHREAD\_CANCELED}.
\item By setting \emph{rval\_ptr} to \code{NULL}, the method does not retrieve
the terminated thread's termination status.
\end{itemize}

\subsection*{11.6 Thread Synchronization}
\begin{itemize}
\item When multiple threads of control share the same memory, we need to make
sure that each thread sees a consistent view of its data.
\item WHen one thread can modify a variable that other threads can read or
modify, we need to synchronize the threads to ensure that they don't use an
invalid value when accessing the variable's memory contents.
\item When one thread modifies a variable, other threads can potentially see
inconsistencies when reading the value of that variable.
\item Thread have to use a lock that will allow only one thread to access the
variable at a time.
\item If the modification is atomic, then there isn't a race condition.
Similarly, if our data always appears to be \emph{sequentially consistent}, then
we need no additional synchronization.
\item Our operations are sequentially consistent when multiple threads can't
observe inconsistencies in our data.
\item In a sequentially consistent environment, we can explain modifications to
our data as a sequential step of operations taken by the running thread.
\end{itemize}

\subsubsection*{11.6.1 Mutexes}
\begin{itemize}
\item A \emph{mutex} is basically a lock that we set (lock) before accessing a
shared resource and release (unlock) when we're done.
\item While it is set, any other thread that tires to set it will block until we
release it.
\item This mutual-exclusion mechanism works only if we design our threads to
follow the same data-access rules.
\item A mutex variable is represented by the \code{pthread\_mutex\_t} data type.
\item Before we can use a mutex variable, we must first initialize it by either
setting it to the constant \code{PTHREAD\_MUTEX\_INITIALIZER} or by calling
\code{pthread\_mutex\_init}.
\item If we allocate the mutex dynamically, then we need to call
\code{pthread\_mutex\_destroy} before free the memory.
\end{itemize}

\code{\#include <pthread.h>}

\code{int pthread\_mutex\_init(pthread\_mutex\_t *restrict \emph{mutex}, const
pthread\_mutexattr\_t *restrict \emph{attr});}

\code{int pthread\_mutex\_destroy(pthread\_mutex\_t *\emph{mutex});}

Both return: 0 if OK, error number on failure

\begin{itemize}
\item To initialize a mutex with the default attributes, we set \emph{attr} to
\code{NULL}.
\item To lock a mutex, we call \code{pthread\_mutex\_lock}. If the mutex is
already locked, the calling thread will block until the mutex is unlocked.
\item To unlock a mutex, we call \code{pthread\_mutex\_unlock}.
\end{itemize}

\code{\#include <pthread.h>}

\code{int pthread\_mutex\_lock(pthread\_mutex\_t *\emph{mutex});}

\code{int pthread\_mutex\_trylock(pthread\_mutex\_t *\emph{mutex});}

\code{int pthread\_mutex\_unlock(pthread\_mutex\_t *\emph{mutex});}

All return: 0 if OK, error number on failure

\begin{itemize}
\item If a thread can't afford to block, it can use
\code{pthread\_mutex\_trylock} to lock the mutex conditionally.
\item If a mutex is unlocked at the time trylock is called, then trylock will
lock the mutex without blocking and return 0.
\end{itemize}

\subsubsection*{11.6.2 Deadlock Avoidance}
\begin{itemize}
\item A thread will deadlock itself if it tries to lock the same mutex twice.
\item Deadlocks can be avoided by carefully controlling the order in which
mutexes are locked.
\item You'll have the potential for a deadlock only when one thread attempts to
lock the mutexes in the opposite order from another thread.
\end{itemize}

\subsubsection*{11.6.3 \code{pthread\_mutex\_timedlock Function}}
\begin{itemize}
\item \code{pthread\_mutex\_timedlock} is equivalent to
\code{pthread\_mutex\_lock}, but if the timeout value is reached,
\code{pthread\_mutex\_timedlock} will return the error code \code{ETIMEDOUT}
without locking the mutex.
\end{itemize}

\code{\#include <pthread.h>}

\code{\#include <time.h>}

\code{int pthread\_mutex\_timedlock(pthread\_mutex\_t *restrict \emph{mutex},
const struct timespec *restrict \emph{tsptr});}

Returns: 0 if OK, error number on failure

\begin{itemize}
\item The timeout specifies how long we are willing to wait in terms of absolute
time.
\item The timeout is represented by the \code{timespec} structure, which
describes time in terms of seconds and nanoseconds.
\end{itemize}

\subsubsection*{11.6.4 Reader-Writer Locks}
\begin{itemize}
\item Three states are possible with a reader-writer lock:
\begin{enumerate}
\item Locked in read mode
\item Locked in write mode
\item Unlocked
\end{enumerate}
\item Only one thread at a time can hold a reader-writer lock in write mode, but
multiple threads can hold a reader-writer lock in read mode at the same time.
\item When a reader-writer lock is write locked, all threads attempting to lock
it block until its unlocked.
\item When a reader-writer lock is read locked, all threads attempting to lock
it in read mode are given access, but any threads attempting to lock it in write
mode block until all the threads have released their read locks.
\item Reader-writer locks are well suited for situations in which data
structures are read more often than they are modified.
\item Reader-writer locks are also called shared-exclusive locks.
\item Must be initialized before use and destroyed before freeing the underlying
memory.
\end{itemize}

\code{\#include <pthread.h>}

\code{int pthread\_rwlock\_init(pthread\_rwlock\_t *restrict \emph{rwlock},
const pthread\_rwlockattr\_t *restrict \emph{attr});}

\code{int pthread\_rwlock\_destroy(pthread\_rwlock\_t *\emph{rwlock});}

Both return: 0 if OK, error on failure

\begin{itemize}
\item Pass null pointer for \emph{attr} if the reader-writer lock is supposed to
have default attributes
\item To lock a reader-writer lock in read mode, we call
\code{pthread\_rwlock\_rdlock}.
\item To write lock a reader-writer lock, we call
\code{pthread\_rwlock\_wrlock}.
\item We unlock the lock by calling \code{pthread\_rwlock\_unlock}.
\end{itemize}

\code{\#include <pthread.h>}

\code{int pthread\_rwlock\_rdlock(pthread\_rwlock\_t *\emph{rwlock});}

\code{int pthread\_rwlock\_wrlock(pthread\_rwlock\_t *\emph{rwlock});}

\code{int pthread\_rwlock\_unlock(pthread\_rwlock\_t *\emph{rwlock});}

All return: 0 if OK, error number on failure

\begin{itemize}
\item We should always check for errors when we call functions that can
potentially fail.
\item However, we do not need to check them if we design our locking properly.
\item We also have the conidtional versions of the reader-writer locking
primitives.
\end{itemize}

\code{\#include <pthread.h>}

\code{int pthread\_rwlock\_tryrdlock(pthread\_rwlock\_t *\emph{rwlock});}

\code{int pthread\_rwlock\_trywrlock(pthread\_rwlock\_t *\emph{rwlock});}

Both return: 0 if OK, error number on failure.

\begin{itemize}
\item When the lock can be acquired, these functions return 0.
\item Otherwise, they return the error \code{EBUSY}.
\end{itemize}

\subsubsection*{11.6.5 Reader-Writer Locking with Timeouts}
\begin{itemize}
\item Provides functions to lock reader-writer loks witha  timeout to give
applications a way to avoid blocking indefinitely.
\end{itemize}

\code{\#include <pthread.h>}

\code{\#include <time.h>}

\code{int pthread\_rwlock\_timedlock(pthread\_rwlock\_t *restrict \emph{rwlock},
const struct timespec *restrict \emph{tsptr});}


\code{int pthread\_rwlock\_timedwrlock(pthread\_rwlock\_t *restrict
\emph{rwlock}, const struct timespec *restrict \emph{tsptr});}

Both return: 0 if OK, error number on failure

\begin{itemize}
\item The \emph{tsptr} argument points to a \code{timespec} structure specifying
the time at which the thread should stop blocking.
\end{itemize}

\subsubsection*{11.6.6 Condition Variables}
\begin{itemize}
\item These synchronization objects provide a place for threads to rendezvous.
\item Protected by a mutex.
\item Can be initialized statically and dynamically
\item Can use \code{pthread\_cond\_destroy} function to deinitialize a condition
variable before freeing its underlying memory.
\end{itemize}

\code{\#include <pthread.h>}

\code{int pthread\_cond\_init(pthread\_cond\_t *restrict \emph{cond}, const
pthread\_condattr\_t *restrict \emph{attr});}

\code{int pthread\_cond\_destroy(pthread\_cond\_t *cond);}

Both return: 0 if OK, error number on failure

\begin{itemize}
\item Again, setting \emph{attr} argument to \code{NULL} gives default
attributes.
\item We use \code{pthread\_cond\_wait} to wait for a condition to be true.
\end{itemize}

\code{\#include <pthread.h>}

\code{int pthread\_cond\_wait(pthread\_cond\_t *restrict \emph{cond},
pthread\_mutex\_t *restrict \emph{mutex});}

\code{int pthread\_cond\_timedwait(pthread\_cond\_t *restrict \emph{cond},
pthread\_mutex\_t *restrict \emph{mutex}, const struct timespec *restrict
\emph{tsptr});}

Both return: 0 if OK, error number on failure

\begin{itemize}
\item The mutex passed to \code{pthread\_cond\_wait} protects the condition.
\item There are two functions to notify threads that a condition has been
satisfied.
\item The \code{pthread\_cond\_signal} function will wake up at least one thread
waiting on a condition.
\item \code{pthread\_cond\_broadcast} function will wake up all threads waiting
on a condition.
\end{itemize}

\code{\#include <pthread.h>}

\code{int pthread\_cond\_signal(pthread\_cond\_t *\emph{cond});}

\code{int pthread\_cond\_broadcast(pthread\_cond\_t *\emph{cond});}

Both return: 0 if OK, error number on failure

\begin{itemize}
\item When we call these functions, we are said to be \emph{signaling} the
thread or condition.
\item We should signal the threads only after changing the state of the
condition.
\end{itemize}

\end{document}
